  
\documentclass[10,conference]{IEEEtran}
\IEEEoverridecommandlockouts
% The preceding line is only needed to identify funding in the first footnote. If that is unneeded, please comment it out.
\usepackage{cite}
\usepackage{amsmath,amssymb,amsfonts}
\usepackage{algorithmic}
\usepackage{graphicx}
\usepackage{textcomp}
\usepackage{hyperref}
\usepackage{lscape}
%\usepackage{xcolor}
\usepackage{float}
\usepackage[table,xcdraw]{xcolor}
\graphicspath{ {pictures/} }
\def\BibTeX{{\rm B\kern-.05em{\sc i\kern-.025em b}\kern-.08em
    T\kern-.1667em\lower.7ex\hbox{E}\kern-.125emX}}
\begin{document}

\title{Cross Network Secure Service Networking}

\author{\IEEEauthorblockN{Georges Ankenmann - 100935237}
\IEEEauthorblockA{\textit{School Of Information and Technology} \\
\textit{Carleton University}\\
Ottawa, Canada \\
georgesankenmann@cmail.carleton.ca}
\and
\IEEEauthorblockN{Fahmida Haque - 101137370}
\IEEEauthorblockA{\textit{School Of Information and Technology} \\
\textit{Carleton University}\\
Ottawa, Canada \\
fahmidahaque@cmail.carleton.ca}
\and
\IEEEauthorblockN{Jeff Bailey - 101020551}
\IEEEauthorblockA{\textit{Department of Systems} \\\textit{and Computer Engineering} \\
\textit{Carleton University}\\
Ottawa, Canada \\
jeffbailey@cmail.carleton.ca}}

\maketitle

\begin{abstract}
Cross network secure service networking is starting to be a more common requirement these days where certain services must be accessible but proper security controls must also be implemented to prevent unauthorized access. Traditionally this can be accomplished by means of a virtual private network (VPN), but this does not always work with services or applications that can't run the required VPN software. Cross network secure service networking allows for services to communicate with each other across networks by establishing an encrypted and authenticated overlay network that allows for secure communication at the service level. This proposal will compare existing frameworks, and propose a new framework that offers security as the preliminary design consideration. 
\end{abstract}

\section{Introduction}
Cross network secure service networking is starting to be a more common requirement these days where certain services must be accessible but proper security controls must also be implemented to prevent unauthorized access.

VPN are present in a modern connected world. One instance is in corporate environments which allows secure access to protected services from untrusted environments, such as the public internet. Secure access to corporate services can be accomplished by means of a client to site or a site to site virtual private network (VPN). Generally with a site to site VPN, the tunnel are usually established by dedicated hardware. With a client to site VPN, the tunnel is usually established by software on a client device\cite{what is a vpn}\cite{how a vpn works}. 

Overlay networks are common in modern data centres where they allow abstraction of certain network functions that can run on top of physical networks\cite{what are network overlays}. A common network overlay is Virtual eXtensible Local Area Network (VXLAN)\cite{vxlan}. This framework allows data centres to offer tenants with their own isolated network domains without deploying additional infrastructure. Network encryption is available for VXLAN by means of IPSec. 

Cross network secure service networking allows for services to communicate with each other across networks by establishing an encrypted and authenticated overlay network that allows for secure communication at the service level. Cross network secure service networking can also be leveraged by legacy applications that don't offer in transit data encryption by means of a bastion host \cite{what is a bastion host} that can functions as a proxy which would take in unencrypted data then pipes it to a service or client over an encrypted tunnel. A common example of such a device are terminal servers\cite{raritan} that connect to supervisory control and data acquisition (SCADA) and legacy equipment that only have serial connection or telent, and offer the user a secure session by means of a SSH session or VPN.

Multiple commercial offerings exist for software applications, notably Consul and Google Compute Engine (GCE) Identity-Aware Proxy (IAP). They offer software that enables cross network secure service networking. Consul can run on almost any operating system, while IAP is Google Cloud specific where it allows a identity aware proxy that allows secure access to non public services.

Consul is software that is majorly used for discovery and configuration for a range of applications and services, it provides an up to date outlook on services in an infrastructure. Consul was mainly built to manage services in a distributed system. One of the main challenges faced by many large scale industries, is the heightened use of distributed systems and service discovery has come as a blessing to the challenges faced by the use of distributed systems. Consul is made user-friendly with the use of a flexible and powerful interface. Consul requires a data plane and it runs with a built-in proxy. All the services discovered by consul in an infrastructure is stored in a single registry so that the services can find each other by storing the information of IP addresses or other location information, this makes consul a centralized service registry. Clients can find services that are registered with consul with the help of DNS or HTTP interfaces which leads to an easy procedure to discover resources\cite{consul}.  

Identity-Aware Proxy (IAP), is a service from Google which provides a “central authorization layer for applications accessed by HTTPS” \cite{iap}. IAP works by only granting users or so-called “members” access when they posses the correct role. This allows for access control policies to be specific to resources and applications\cite{iap}. IAP works by intercepting web requests, authenticating the user making the request, and only granting access when the user is in fact authorized. This technique of implementing authentication and authorization remotely removes the need for a VPN \cite{iap_b}. Google’s goal with IAP is to create a method that users may connect to untrusted networks and access services without needing a VPN \cite{iap_c} – this directly mitigates the issues surrounding VPN usage, such as reduced speed. As well, users of VPN’s must either have full access or no access to the network, and service granularity is extremely limited – IAP solves this by providing authentication on an application by application basis.

As the previously mentioned Consul run in software while Google IAP runs only within Google Cloud. Consul is not designed to operate with physical equipment. 

In this project we will be experimenting with available products along with a custom solution that we will design that will meet our requirements. We will be using Docker Engine with Docker Compose scripts to experiment with the software along with creating a consistent setup for a more accurate comparison. 
\section{Problem Statement}
The issue you have with overlay networks is that it must be implemented at the network level, it is not implemented at the service level. This is where cross network secure service networking comes into play. This proposal will compare existing frameworks, and propose a new framework that offers security as the preliminary design consideration. 

\section{Objective}
The primary objective is that we design a system that allows for encryption of data without changing the client or server implementation.


\section{Metrics}
One of the first tasks we had to complete was to determine what metrics we were looking for to determine the "best" solution. We settled on the following metrics.
\begin{itemize}
\item Latency
\item Throughput
\item Resource utilization of host and containers
\item Packets dropped
\end{itemize}

\section{Software}
After many hours of research we settled on the following software.
\subsection{Backend Software (out of scope of this project)}
For this project we used some software in the backend to make our lives much easier to analyze the data and provide meaningful results. The following list of software is not within the scope of this project as it does not actually relate to the problem statement, but it offers the possibility to accomplish research relating to the problem statement. 
\begin{itemize}
\item Kibana
\item Elasticsearch
\item Mosquitto MQTT Broker
\end{itemize}
\subsection{Lab Software (in scope of this project)}
For this project we used some software in our lab that we determined that would be best for the project. 
\begin{itemize}
\item Docker Engine
\item Docker Compose
\item Shadowsocks
\item stunnel
\item Wireguard
\item Wireshark
\item NGINX
\end{itemize}
\section{Lab Setup}
We have setup a lab environment that is consistent and easy to use. We used Docker Engine with Docker Compose scripts to allow for a easy lab setup. A Github repository has been provided with all our research and scripts\cite{github_repo}.

After installing Docker Engine and Docker Compose on our machines, we then started to experiment with designing scenarios on how to best implement the software we found. 
\subsection{Base Lab Setup}
\subsection{Wireguard Lab Setup}
Wireguard is a lightweight VPN intended for the Linux kernel, but is now available for Windows and macOS as well. It is a performance-oriented VPN that functions using a concept called Cryptokey Routing [1j]. This is a technique of assigning public keys to a list of allowable tunnel IP addresses, where each network interface is given a private key and a list of peers – keys are shared similarly to how SSH public keys are shared. Configuring Wireguard VPN’s is a relatively simple task, once Wireguard is installed it simply needs to be configured on each device, where one device acts as the server and the other devices respectively operate as clients. Each device’s configuration file contains the private key, public key of its peers, its listen port, and a list of allowed IP addresses. Wireguard can use IPv4 and IPv6 addresses [1j]. 

Like most VPN’s Wireguard works on the network layer (layer 3) of the OSI model whereas the HTTP client-server and MQTT client-server programs of this environment operate on the application layer (layer 7) of the OSI model. As such, it must be interfaced properly to encrypt packets leaving the HTTP and MQTT services. The goal of this experimental environment is the same as in the other environments, encrypt the traffic coming from a service without changing the service itself, thus it is important not to alter the HTTP and MQTT containers. To connect the client and server through Wireguard an intermediate step is necessary connect from layer 7 down to layer 3. To do so, NGINX is used. NGINX is an open-source software that can be used for a multitude of web server-related use cases [2j]. In the case of this environment, it is used to stream data from the layer 7 applications, HTTP and MQTT, to the layer 3 Wireguard VPN. 

To create a secure tunnel between the services, the client and server which are docker containers, Wireguard is implemented as a proxy container to connect them over the untrusted network. Then an NGINX container that streams and redirects incoming data to the correct ports is placed between the service containers and their respective Wireguard proxy containers for encryption (see figure 1j).
\subsection{Stunnel Lab Setup}
\subsection{Shadowsocks Lab Setup}
\section{Results and Analysis}
To gather metrics, searches must be performed in the Elasticsearch database in order to extract the data. Kibana is used to visualize the searches. Saved objects are available on the project's GitHub\cite{github_repo} to replicate our searches. The Kibana saved objects loads saved searches and visualizations that automatically colour results. The table \ref{tab:kibana-colour} depicts the mapping of metrics to colour scheme used to automatically colour some of the data used in the graphs for this project. We determined these values after thinking what we believe to be "\textit{reasonable}". The colour is just to make the graphs pretty and easy to quickly analyze, the do not impact the results in anyway.

For each lab setup, we would run the lab for a minimum of 25 minutes. Metrics are calculated on the latest 15 minutes of data based on the time of ingest into Elasticsearch. A table with our overall results is included in this report, see table \ref{tab:results} for details. If the reader wishes to replicate our project, Docker Compose scripts are available in the project's GitHub repository \cite{github_repo}.

\subsection{Base Results}
\begin{figure}[H]
\includegraphics[width=9cm]{lab-base-results}
\centering
\caption{Kibana screenshot of the results from the base lab setup}
\label{pic:base-results}
\end{figure}
Over the last 15 minutes of data from this setup, we found an average latency of 9.326ms and 1.832ms for HTTP and MQTT respectively. This setup averaged 0.849\% in terms of CPU usage and 0.791\% in terms of memory usage. It should be noted that this setup only uses 9 containers. HTTP and MQTT data in transit are is \textbf{not} encrypted in this lab.

\subsection{Wireguard Results}
\begin{figure}[H]
\includegraphics[width=9cm]{lab-wireguard-results}
\centering
\caption{Kibana screenshot of the results from the Wireguard lab setup}
\label{pic:wireguard-results}
\end{figure}
Over the most recent 15 minutes of run time for this environment, we found an average latency of 10.810ms and 4.002ms for HTTP and MQTT respectively. For CPU and memory usage, the entire environment averaged 0.542\% and 0.465\% respectively. This is a 1.484ms and 2.170ms increase in latency for HTTP and MQTT respectively over the unencrypted base environment. 

The interesting part of these results is the fact that in figure \ref{pic:wireguard-results} we observed a slight decrease of 0.302\% and 0.326\% in CPU and memory usage respectively. While the designers that developed Wireguard claim that Wireguard "\textit{[...]to be faster, simpler, leaner, and more useful than IPsec[...]}"\cite{wireguard}. While we don't refute these claims, our current working hypothesis is that this is due to the fact that in this environment we have 13 containers, and the CPU and memory is averaged over 13 containers. Further work will be required to determine how to gather the data required to determine the exact source of these interesting results. HTTP and MQTT data in transit are encrypted in this lab.
\subsection{Stunnel Results}
\begin{figure}[H]
\includegraphics[width=9cm]{lab-stunnel-results}
\centering
\caption{Kibana screenshot of the results from the Stunnel lab setup}
\label{pic:stunnel-results}
\end{figure}
Over the most recent 15 minutes of run time for this environment, we found an average latency of 16.943ms and 3.184ms for HTTP and MQTT respectively. For CPU and memory usage, the entire environment averaged 4.87\% and 0.468\% respectively. This is a 7.617ms and 1.352ms increase in latency for HTTP and MQTT respectively over the unencrypted base environment. 

To us, the most interesting part of these results is the high CPU usage. This lab setup has consumed significantly more CPU usage when comparing to the other setups, over 4\% above base. It is the highest recorded ever the entire project. Our current working hypothesis is that the stunnel process is really CPU intensive, more work will be required to chase down the cause of this. HTTP and MQTT data in transit are is encrypted in this lab.
\subsection{Shadowsocks Results}
\begin{figure}[H]
\includegraphics[width=9cm]{lab-shadowsocks-results}
\centering
\caption{Kibana screenshot of the results from the Shadowsocks lab setup}
\label{pic:shadowsocks-results}
\end{figure}
Shadowsocks \cite{shadowsocks}, which is a SOCKS5 proxy \cite{socks5}, did not fulfill our project objective where we had do find or design a system that would allow for encryption of data without changing the client or server implementation. In the case of Shadowsocks, we had to modify the HTTP client implementation to use the SOCKS5 proxy. 

We had an issue with the MQTT intance for this lab setup. The MQTT Python library does not support SOCKS5, therefore we were not able to gather metrics for MQTT via a Shadowsocks implementation.  

Even if with these limitation, we opted to see what metrics would be generated by a modified HTTP setup. Over the most recent 15 minutes of run time for this environment, we found an average latency of 9.679ms for HTTP. For CPU and memory usage, the entire environment averaged 0.534\% and 0.658\% respectively. This is a 0.353ms latency increase for HTTP over the unencrypted base environment, which is certainly a respectable time difference considering that there are only 8 containers and the HTTP data is encrypted. 
\section{Literature Review}
\section{Conclusion}
\section{Future Work}

\section{Appendix}
\begin{table}[H]
\centering
\begin{tabular}{ccc}
\hline
\multicolumn{3}{|c|}{\textbf{Kibana Colour Boxes (CPU and Memory)}} \\ \hline
\multicolumn{1}{|c|}{Colour} & \multicolumn{1}{c|}{Start} & \multicolumn{1}{c|}{End} \\ \hline
\multicolumn{1}{|c|}{Green} & \multicolumn{1}{c|}{0} & \multicolumn{1}{c|}{1} \\ \hline
\multicolumn{1}{|c|}{|} & \multicolumn{1}{c|}{\textgreater{}1} & \multicolumn{1}{c|}{5} \\ \hline
\multicolumn{1}{|c|}{Red} & \multicolumn{1}{c|}{\textgreater{}10} & \multicolumn{1}{c|}{100} \\ \hline
\multicolumn{1}{l}{} & \multicolumn{1}{l}{} & \multicolumn{1}{l}{} \\ \hline
\multicolumn{1}{|c|}{\textbf{Kibana Colour Boxes (Latency, in ms)}} & \multicolumn{1}{c|}{Start} & \multicolumn{1}{c|}{End} \\ \hline
\multicolumn{1}{|c|}{Green} & \multicolumn{1}{c|}{0} & \multicolumn{1}{c|}{} \\ \hline
\multicolumn{1}{|c|}{|} & \multicolumn{1}{c|}{1} & \multicolumn{1}{c|}{5} \\ \hline
\multicolumn{1}{|c|}{|} & \multicolumn{1}{c|}{\textgreater{}5} & \multicolumn{1}{c|}{10} \\ \hline
\multicolumn{1}{|c|}{|} & \multicolumn{1}{c|}{\textgreater{}10} & \multicolumn{1}{c|}{100} \\ \hline
\multicolumn{1}{|c|}{Red} & \multicolumn{1}{c|}{\textgreater{}100} & \multicolumn{1}{c|}{20000} \\ \hline
\multicolumn{1}{l}{} & \multicolumn{1}{l}{} & \multicolumn{1}{l}{} \\ \hline
\multicolumn{1}{|c|}{\textbf{\begin{tabular}[c]{@{}c@{}}Average Packet Drops and Errors\\ (occurrences)\end{tabular}}} & \multicolumn{1}{c|}{Start} & \multicolumn{1}{c|}{End} \\ \hline
\multicolumn{1}{|c|}{Green} & \multicolumn{1}{c|}{0} & \multicolumn{1}{c|}{5} \\ \hline
\multicolumn{1}{|c|}{Red} & \multicolumn{1}{c|}{\textgreater{}5} & \multicolumn{1}{c|}{99999999} \\ \hline
\end{tabular}
\caption{Kibana Colour Boxes}
\label{tab:kibana-colour}
\end{table}
\begin{landscape}
\begin{table}[]
\centering
\begin{tabular}{|c|c|c|c|c|c|c|c|c|c|c|}
\hline
 & \multicolumn{3}{c|}{\textbf{HTTP Server Latency}} & \multicolumn{3}{c|}{\textbf{MQTT Server Latency}} & \multicolumn{4}{c|}{} \\ \hline
Scenario & \begin{tabular}[c]{@{}c@{}}Average\\ (ms)\end{tabular} & \begin{tabular}[c]{@{}c@{}}Minimum\\ (ms)\end{tabular} & \begin{tabular}[c]{@{}c@{}}Maximum\\ (ms)\end{tabular} & \begin{tabular}[c]{@{}c@{}}Average \\ (ms)\end{tabular} & \begin{tabular}[c]{@{}c@{}}Minimum\\ (ms)\end{tabular} & \begin{tabular}[c]{@{}c@{}}Maximum\\ (ms)\end{tabular} & \begin{tabular}[c]{@{}c@{}}Average\\ CPU\\ (\%)\end{tabular} & \begin{tabular}[c]{@{}c@{}}Average\\ Memory\\ (\%)\end{tabular} & \begin{tabular}[c]{@{}c@{}}\#\\ Containers\end{tabular} & \begin{tabular}[c]{@{}c@{}}\#\\ Packet Drops\end{tabular} \\ \hline
\rowcolor[HTML]{FFCE93} 
Base & 9.326 & 2 & 16 & 1.832 & 0 & 6 & 0.849 & 0.791 & 9 & 0 \\ \hline
\rowcolor[HTML]{FFCCC9} 
Shadowsocks & 9.679 & 3 & 24 & \multicolumn{3}{c|}{\cellcolor[HTML]{FFCCC9}N/A} & 0.534 & 0.658 & 11 & 0 \\ \hline
Wireguard & \cellcolor[HTML]{9AFF99}10.81 & \cellcolor[HTML]{9AFF99}3 & \cellcolor[HTML]{9AFF99}19 & 4.002 & 1 & 12 & \cellcolor[HTML]{9AFF99}0.547 & \cellcolor[HTML]{9AFF99}0.465 & 13 & \cellcolor[HTML]{9AFF99}0 \\ \hline
Stunnel & 16.943 & 4 & 56 & \cellcolor[HTML]{9AFF99}3.184 & \cellcolor[HTML]{9AFF99}1 & \cellcolor[HTML]{9AFF99}21 & 4.87 & 0.486 & 13 & \cellcolor[HTML]{9AFF99}0 \\ \hline
\end{tabular}
\caption{Results table.}
\label{tab:results}
\end{table}
\end{landscape}
\begin{thebibliography}{00}
\bibitem{what is a vpn} Norton, \textit{What is a VPN?}, Norton, 2020. Accessed on: September 20, 2020. [Online]. Available: \url{https://us.norton.com/internetsecurity-privacy-what-is-a-vpn.html}

\bibitem{raritan} Raritan, \textit{Serial Console Server - Your Next Generation Solution}, Raritan, 2020. Accessed on: September 20, 2020. [Online]. Available: \url{https://www.raritan.com/products/kvm-serial/serial-console-servers/serial-over-ip-console-server}

\bibitem{how a vpn works} J. Tyson, C. Pollette, S. Crawford, \textit{How a VPN (Virtual Private Network) Works}, How Stuff Works, 2019. Accessed on: September 20, 2020. [Online]. Available: \url{https://computer.howstuffworks.com/vpn.htm}

\bibitem{what is a bastion host} J. Kozlowicz, \textit{What’s a Jumpbox or Bastion Host, Anyway?}, Green House Data, 2019. Accessed on: September 20, 2020. [Online]. Available: \url{https://www.greenhousedata.com/blog/whats-a-jumpbox-or-bastion-host-anyway}

\bibitem{what are network overlays} SDxCentral, \textit{What are Network Overlays}, SDxCentral, 2015. Accessed on: September 20, 2020. [Online]. Available: \url{https://www.sdxcentral.com/networking/virtualization/definitions/get-on-top-of-network-overlays/}

\bibitem{vxlan} IETF, \textit{Virtual eXtensible Local Area Network (VXLAN): A Framework
   for Overlaying Virtualized Layer 2 Networks over Layer 3 Networks}, IETF, 2014. Accessed on: September 20, 2020. [Online]. Available: \url{https://tools.ietf.org/html/rfc7348}
 
\bibitem{consul} Consul, \textit{Consul by HashiCorp}, HashiCorp, 2020. Accessed on: September 20, 2020. [Online] \url{https://www.consul.io/}

\bibitem{iap} Google Cloud, \textit{Identity-Aware Proxy (IAP)}, Google, 2020. Accessed on: September 20, 2020. [Online] Available: \url{https://cloud.google.com/iap}
\bibitem{iap_b} Google Cloud, \textit{Identity-Aware Proxy: Documentatio }, Google, 2020. Accessed on: October 4, 2020. [Online] Available: \url{https://codelabs.developers.google.com/codelabs/user-auth-with-iap/index.html?index=..%2F..index#0}
\bibitem{iap_c} T. Treat, \textit{API Authentication with GCP Identity-Aware Proxy}, Medium.com, January 25th, 2019. Accessed: October 4, 2020. [Online]. Available: \url{https://blog.realkinetic.com/api-authentication-with-gcp-identity-aware-proxy-3a4147b30770}
\bibitem{github_repo} G. Ankenmann, J. Bailey, F. Haque \textit{ITEC5102 Project Repository}, Github.com. [Online]. Available: \url{https://github.com/devagent42/ITEC5102F-Project}
\bibitem{wireguard} J. A. Donenfeld, \textit{Wireguard}, Wireguard, 2020. Accessed: October 4, 2020. [Online]. Available: \url{https://www.wireguard.com/}

\bibitem{stunnel} Stunnel, \textit{Stunnel}, Stunnel, 2020. Accessed: October 4, 2020. [Online]. Available: \url{https://www.stunnel.org/}

\bibitem{shadowsocks} Shadowsocks, \textit{Shadowsocks}, Shadowsocks, 2020. Accessed: October 4, 2020. [Online]. Available: \url{https://shadowsocks.org/en/index.html}

\bibitem{socks5} A. Patil, \textit{SOCKS Proxy Primer: What Is SOCKs5 and Why Should You Use It?}, Shadowsocks, September 27, 2019. Accessed: October 4, 2020. [Online]. Available: \url{https://securityintelligence.com/posts/socks-proxy-primer-what-is-socks5-and-why-should-you-use-it/}
\end{thebibliography}

\end{document}